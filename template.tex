\documentclass{lcc}

\materia{Materia X}
\logo{unr.jpg}
\practica{1 - Introducción a ...}

% \commentstrue
% \soluciones

\begin{document}
\maketitle

Este es un ejemplo de una práctica. Dentro hay algunos if que pueden
encenderse para mostrar las soluciones de los problemas.
\guido{Por ejemplo, este es un comentario}

\begin{problemaOff}[(Filósofos Comensales, Dijkstra)]
    Esto es un problema oculto. Cambiar \texttt{problemaOff} a
    \texttt{problema} para mostrarlo.
\end{problemaOff}

\begin{problema}
    Esto es un problema.
    \begin{solucion}
        Acá está la solución.
    \end{solucion}
    \begin{solucion}[variante 2]
        Acá hay otra solucion.
    \end{solucion}
\end{problema}

\begin{problema}[(díficil)]
    Esto es un problema con un tag opcional.
\end{problema}

\begin{problema}
    Este problema tiene un fragmento de código C.
    \begin{C}
int main()
{
    return 0;
}
    \end{C}
    Y tambien algo de codigo C inline: \cc{free(NULL)}.
\end{problema}

\begin{problema}
    También puede incluirse código Bash.
    \begin{Bash}
a=0
while [ $a -le 100 ]; do
    echo $a
    a=$((a+1))
done
    \end{Bash}
\end{problema}

\end{document}
