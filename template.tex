\documentclass[practica]{lcc}

\codigo{R-999}
\materia{Magia Negra}
\num{1}
\titulo{Introducción a las Artes Ocultas}

\commentstrue
\soluciones

\def\tagproblema{Ej. }

\begin{document}

\maketitle

Este es un ejemplo de una práctica. Dentro hay algunos if que pueden
encenderse para mostrar las soluciones de los problemas.
\guido{Por ejemplo, este es un comentario}

\begin{problema}
    El siguiente código computa el valor $\frac{1}{\sqrt{x}}$. Explique
    su funcionamiento.
    \begin{C}
float Q_rsqrt(float number)
{
    long i;
    float x2, y;
    const float threehalfs = 1.5F;

    x2 = number * 0.5F;
    y  = number;
    i  = * (long *) &y;                   // evil floating point bit level hacking
    i  = 0x5f3759df - (i >> 1);           // what the fuck?
    y  = * (float *) &i;
    y  = y * (threehalfs - (x2 * y * y)); // 1st iteration
    y  = y * (threehalfs - (x2 * y * y)); // 2nd iteration, this can be removed

    return y;
}
    \end{C}
\end{problema}

\begin{problemaOff}[(Filósofos Comensales, Dijkstra)]
    Esto es un problema oculto. Cambiar \texttt{problemaOff} a
    \texttt{problema} para mostrarlo.
\end{problemaOff}

\begin{problema}
    Esto es un problema.
    \begin{solucion}
        Acá está la solución.
    \end{solucion}
    \begin{solucion}[variante 2]
        Acá hay otra solucion.
    \end{solucion}
\end{problema}

\begin{problema}[(difícil)]
    Este es un problema con un tag opcional.
\end{problema}

\begin{problema}[\unskip*]
    Este es un problema con una estrella.
\end{problema}

\begin{problema}
    Este problema tiene un fragmento de código C.
    \begin{C}
int main()
{
    return 0;
}
    \end{C}
    Y tambien algo de codigo C inline: \cc{return (v&(v-1)) == 0;}.
\end{problema}

\begin{problema}
    También puede incluirse código Bash.
    \begin{Bash}
a=0
while [ $a -le 100 ]; do
    echo $a
    a=$((a+1))
done
    \end{Bash}
    O también inline: \bash{:()\{ :|:& \};:}.
\end{problema}

\end{document}
